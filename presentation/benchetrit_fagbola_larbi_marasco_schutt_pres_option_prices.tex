\documentclass{beamer}
\usetheme{Madrid}
\usecolortheme{default}

% --- Presentation info ---
% IMPORTANT: Save as benchetrit_fagbola_larbi_marasco_schutt_option_prices.tex
\title[Option-implied moments]{Recovering probability moments from inflation option prices}
\author[Group 1]{Ethan Benchetrit, Evelyn Fagbola, Hanane Larbi, \\ Lorenzo Marasco, Eduardo Schutt}
\institute[FTD 2025-26]{Quantitative Methods in Finance - Prof. Eric Vansteenberghe}
\date{January 12, 2026}

\begin{document}

% --- Slide 1: Title ---
\begin{frame}
    \titlepage
\end{frame}

% --- Slide 2: Project scope and objectives ---
\begin{frame}{Project scope and objectives}
    \begin{itemize}
        \item \textbf{Dataset:} 1-year zero-coupon inflation caps, floors, and swaps (EU and US areas).
        \item \textbf{Main goal:} Recovering implied moments (mean, variance, skewness, kurtosis) from risk-neutral distributions.
        \item \textbf{Methodology:} Focus on identification and stability across different dates.
        \item \textbf{Scientific boundary:} This is a recovery project, not a forecasting exercise.
    \end{itemize}
\end{frame}

% --- Slide 3: Literature review: classification per task ---
\begin{frame}{Literature review: classification per task}
    \begin{enumerate}
        \item \textbf{Task 1.3.1: Prices $\to$ PDF (Nonparametric)}
        \begin{itemize}
            \item Breeden-Litzenberger (1978): second derivative method ($\sim$3,315 cites).
            \item Jackwerth-Rubinstein (1996): constrained recovery and smoothing ($\sim$1,614 cites).
        \end{itemize}
        \item \textbf{Task 1.3.2: PDF $\to$ moments (Analytical/Numeric)}
        \begin{itemize}
            \item Numerical quadrature: standard integration (trapezoidal rule).
            \item Analytic formulas: moments for parametric families like Lognormal.
        \end{itemize}
        \item \textbf{Task 1.3.3: Prices $\to$ moments (Direct)}
        \begin{itemize}
            \item Bakshi-Kapadia-Madan (2003): direct extraction via replication ($\sim$1,858 cites).
        \end{itemize}
    \end{enumerate}
\end{frame}

% --- Slide 4: Data processing and quality gates ---
\begin{frame}{Data processing and quality gates}
    \begin{itemize}
        \item \textbf{Scaling:} Prices divided by 100 to get price-per-1 units.
        \item \textbf{Reference strikes:} ATM gross strike $K^*$ defined by 1y inflation swap rates.
        \item \textbf{No-arbitrage filters:}
        \begin{itemize}
            \item Put-call parity residual check ($PARITY\_EPS$ gate).
            \item Shape constraints: enforcing monotonicity and convexity.
            \item Strike span diagnostics: checking $K_{min}$ and $K_{max}$ coverage.
        \end{itemize}
    \end{itemize}
\end{frame}

% --- Slide 5: Method 1: Breeden-Litzenberger (BL) ---
% --- Slide 5: Method 1: Breeden-Litzenberger (BL) ---
\begin{frame}{Method 1: Breeden-Litzenberger (1978) (1)}
    \begin{block}{1. Data cleaning and quality control}
        \begin{itemize}
            \item \textbf{OTM selection}: Using Out-of-the-Money options to capture the most reliable information.
            \item \textbf{Parity check}: Verifying the Put-Call parity to ensure market consistency.
            \item \textbf{Data alignment}: Converting Floor prices into Call equivalents:
            \item[] \centering $C = P + B(F - K)$
        \end{itemize}
    \end{block}

    \begin{block}{2. Price curve construction}
        \begin{itemize}
            \item \textbf{Cubic splines}: Creating a smooth curve from discrete market points.
            \item \textbf{Financial constraints}: 
                \begin{itemize}
                    \item Downward slope: Prices must decrease as the strike increases.
                    \item Convexity: The curve must be convex to avoid arbitrage opportunities.
                \end{itemize}
        \end{itemize}
    \end{block}
\end{frame}


\begin{frame}{Method 1: Breeden-Litzenberger (1978) (2)}
    \begin{block}{3. The Breeden-Litzenberger method}
        The probability density $f(S_T)$ is calculated using the second derivative of the Call price curve:
        \vspace{0.3cm}
        \[ f(S_T) = e^{rT} \frac{\partial^2 C}{\partial K^2} \]
        \vspace{0.2cm}
        \textit{Concept}: The curvature of the price curve reveals the market's inflation expectations.
    \end{block}

    \begin{block}{4. Market insights}
        \begin{itemize}
            \item \textbf{Skewness}: Measures if the market fears high inflation (right side) or low inflation (left side).
            \item \textbf{Kurtosis}: Indicates the risk of extreme economic shocks (fat tails).
            \item \textbf{Validation}: The total sum of probabilities must be equal to 1.
        \end{itemize}
    \end{block}
\end{frame}

% --- Slide 6: Method 2: Direct BKM extraction ---
% --- Slide 6: Method 2: Direct BKM extraction ---
\begin{frame}{Method 2 : Bakshi-Kapadia-Madan (2003)}
    \begin{block}{1. From Discrete Prices to Moments}
        The BKM method uses \textbf{Numerical Integration} (Trapezoidal rule) to recover moments directly:
        \[ m_n = K_0^n + \int_{0}^{K_0} g''(K) P(K) dK + \int_{K_0}^{\infty} g''(K) C(K) dK \]
        \begin{itemize}
            \item \textbf{$g''(K)$}: Quadrature weights (2, $6K$, or $12K^2$).
            \item \textbf{$P(K), C(K)$}: Undiscounted Put and Call prices.
        \end{itemize}
    \end{block}

    \begin{block}{2. Numerical Illustration}
        Example for a Call at $K = 1.05$ (Price = $0.001$):
        \begin{itemize}
            \item \textbf{Variance contribution}: $0.001 \times 2 = 0.002$.
            \item \textbf{Kurtosis contribution}: $0.001 \times (12 \times 1.05^2) \approx 0.013$.
        \end{itemize}
    \end{block}
\end{frame}

% --- Slide 7: Parametric benchmarks ---
\begin{frame}{Parametric Benchmarks \& Model Validation}

    \begin{block}{Lognormal Calibration}
        \begin{itemize}
            \item \textbf{Methodology:} Fitting $(\mu, \sigma)$ to Out-of-the-Money (OTM) caps and floors by minimizing pricing RMSE.
            \item \textbf{Martingale constraint:} Use of a penalty anchor to ensure the mean is consistent with the \textbf{forward swap rate} ($K^*$).
            \item \textbf{Tail stability:} Preferred for variance recovery in sparse-strike environments as it provides structural regularization for the tails.
        \end{itemize}
    \end{block}

    \vfill

    \begin{block}{Normal Fit}
        \begin{itemize}
            \item \textbf{Process:} Gaussian fit performed on the density proxy recovered via the Breeden-Litzenberger (BL) method.
            \item \textbf{Role:} Strictly diagnostic; helps identify \textbf{truncation bias} and sensitivity to boundary-mass thresholds.
            \item \textbf{Limitations:} Used to flag positivity mismatches and artifacts caused by limited strike ranges.
        \end{itemize}
    \end{block}
\end{frame}

% --- Slide 8: Empirical results: implied mean ---
\begin{frame}{Empirical results: implied mean (1st moment)}
    \begin{itemize}
        \item \textbf{Method convergence:} High agreement across BL, BKM, and Lognormal.
        \item \textbf{Identification:} Results are strongly anchored by liquid ATM info and swap rates.
    \end{itemize}
    \begin{center}
        \includegraphics[width=0.9\textwidth]{mean_comparison_EU.png}
    \end{center}
\end{frame}

% --- Slide 9: Empirical results: implied variance ---
\begin{frame}{Empirical results: implied variance (2nd moment)}
    \begin{itemize}
        \item \textbf{Divergence diagnosis:} Gaps appear when strike coverage is limited.
        \item \textbf{BKM bias:} Downward bias due to missing far-OTM strikes in the tails.
        \item \textbf{Context:} Methods tend to converge during high-liquidity stress events.
    \end{itemize}
    \begin{center}
        \includegraphics[width=0.9\textwidth]{variance_comparison_EU.png}
    \end{center}
\end{frame}

% --- Slide 10: Skewness, kurtosis, and stability ---
\begin{frame}{Skewness, kurtosis, and stability}
    \begin{itemize}
        \item \textbf{Instability factors:} High sensitivity to tail balance and sparse quotes.
        \item \textbf{Kurtosis behavior:} Most volatile moment; depends heavily on extreme strikes.
        \item \textbf{Our take:} Higher moments serve as qualitative risk indicators rather than exact levels.
    \end{itemize}
\end{frame}

\begin{frame}{Diagnostic findings on variance gaps}
    \footnotesize
    \begin{itemize}
        \item \textbf{Large variance gaps:} BL variance is often 2--5.7$\times$ higher than BKM on the EU sample.
        \item \textbf{Edge-mass analysis:} divergences are not explained by boundary issues; BL density is not boundary-dominated on worst dates.
        \item \textbf{Parity consistency:} overlapping-strike residuals remain very small ($10^{-4}$ to $10^{-3}$) during these divergence periods.
        \item \textbf{Strike centrality:} $K^*$ remains well-positioned near the center (~0.38--0.57), not at the edges of the strike range.
        \item \textbf{Model parameters:} lognormal $\sigma$ bounds are not binding on the dates with the largest gaps.
        \item \textbf{Common-support impact:} gaps between BL and BKM remain high even when the comparison is restricted to overlapping strikes.
    \end{itemize}
    \begin{figure}
    \centering
    \includegraphics[width=0.6\linewidth]{image.png}
    \label{fig:placeholder}
\end{figure}
\end{frame}

% --- Slide 11: Final conclusion and takeaways ---
\begin{frame}{Final conclusion and takeaways}
    \begin{itemize}
        \item \textbf{Mean:} Robust identification across methods using swap anchoring.
        \item \textbf{Variance:} Tail-regularized methods (Lognormal/Smoothed BL) are better for sparse data.
        \item \textbf{Direct BKM:} Best interpreted as a lower-bound for variance in illiquid periods.
        \item \textbf{Final word:} Careful strike-span monitoring is essential for reliable moment recovery.
    \end{itemize}
\end{frame}

\end{document}


